%%%%%%%%%%%%  Generated using docx2latex.com  %%%%%%%%%%%%%%

%%%%%%%%%%%%  v2.0.0-beta  %%%%%%%%%%%%%%

\documentclass[12pt]{report}
\usepackage{amsmath}
\usepackage{latexsym}
\usepackage{amsfonts}
\usepackage[normalem]{ulem}
\usepackage{array}
\usepackage{amssymb}
\usepackage{graphicx}
\usepackage[backend=biber,
style=numeric,
sorting=none,
isbn=false,
doi=false,
url=false,
]{biblatex}\addbibresource{bibliography.bib}

\usepackage{subfig}
\usepackage{wrapfig}
\usepackage{wasysym}
\usepackage{enumitem}
\usepackage{adjustbox}
\usepackage{ragged2e}
\usepackage[svgnames,table]{xcolor}
\usepackage{tikz}
\usepackage{longtable}
\usepackage{changepage}
\usepackage{setspace}
\usepackage{hhline}
\usepackage{multicol}
\usepackage{tabto}
\usepackage{float}
\usepackage{multirow}
\usepackage{makecell}
\usepackage{fancyhdr}
\usepackage[toc,page]{appendix}
\usepackage[hidelinks]{hyperref}
\usetikzlibrary{shapes.symbols,shapes.geometric,shadows,arrows.meta}
\tikzset{>={Latex[width=1.5mm,length=2mm]}}
\usepackage{flowchart}\usepackage[paperheight=11.69in,paperwidth=8.27in,left=0.98in,right=0.98in,top=0.98in,bottom=0.98in,headheight=1in]{geometry}
\usepackage[utf8]{inputenc}
\usepackage[T1]{fontenc}
\TabPositions{0.49in,0.98in,1.47in,1.96in,2.45in,2.94in,3.43in,3.92in,4.41in,4.9in,5.39in,5.88in,}

\urlstyle{same}


 %%%%%%%%%%%%  Set Depths for Sections  %%%%%%%%%%%%%%

% 1) Section
% 1.1) SubSection
% 1.1.1) SubSubSection
% 1.1.1.1) Paragraph
% 1.1.1.1.1) Subparagraph


\setcounter{tocdepth}{5}
\setcounter{secnumdepth}{5}


 %%%%%%%%%%%%  Set Depths for Nested Lists created by \begin{enumerate}  %%%%%%%%%%%%%%


\setlistdepth{9}
\renewlist{enumerate}{enumerate}{9}
		\setlist[enumerate,1]{label=\arabic*)}
		\setlist[enumerate,2]{label=\alph*)}
		\setlist[enumerate,3]{label=(\roman*)}
		\setlist[enumerate,4]{label=(\arabic*)}
		\setlist[enumerate,5]{label=(\Alph*)}
		\setlist[enumerate,6]{label=(\Roman*)}
		\setlist[enumerate,7]{label=\arabic*}
		\setlist[enumerate,8]{label=\alph*}
		\setlist[enumerate,9]{label=\roman*}

\renewlist{itemize}{itemize}{9}
		\setlist[itemize]{label=$\cdot$}
		\setlist[itemize,1]{label=\textbullet}
		\setlist[itemize,2]{label=$\circ$}
		\setlist[itemize,3]{label=$\ast$}
		\setlist[itemize,4]{label=$\dagger$}
		\setlist[itemize,5]{label=$\triangleright$}
		\setlist[itemize,6]{label=$\bigstar$}
		\setlist[itemize,7]{label=$\blacklozenge$}
		\setlist[itemize,8]{label=$\prime$}



 %%%%%%%%%%%%  Header here  %%%%%%%%%%%%%%


\pagestyle{fancy}
\fancyhf{}
\chead{ 
\vspace{\baselineskip}
}
\renewcommand{\headrulewidth}{0pt}
\setlength{\topsep}{0pt}\setlength{\parskip}{9.96pt}
\setlength{\parindent}{0pt}

 %%%%%%%%%%%%  This sets linespacing (verticle gap between Lines) Default=1 %%%%%%%%%%%%%%


\renewcommand{\arraystretch}{1.3}


%%%%%%%%%%%%%%%%%%%% Document code starts here %%%%%%%%%%%%%%%%%%%%



\begin{document}

\vspace{\baselineskip}
\setlength{\parskip}{0.0pt}


%%%%%%%%%%%%%%%%%%%% Figure/Image No: 1 starts here %%%%%%%%%%%%%%%%%%%%

\begin{figure}[H]
\advance\leftskip 3.98in
\begin{center}
    
\includegraphics[width=2.51in,height=1.51in]{./media/image3.png}
\end{center}
\end{figure}


%%%%%%%%%%%%%%%%%%%% Figure/Image No: 1 Ends here %%%%%%%%%%%%%%%%%%%%

\par




\vspace{\baselineskip}

\vspace{\baselineskip}
\setlength{\parskip}{9.96pt}

\vspace{\baselineskip}

\vspace{\baselineskip}

\vspace{\baselineskip}
\setlength{\parskip}{0.0pt}
\begin{Center}
 {\fontsize{28pt}{33.6pt}\selectfont \textbf{\textcolor[HTML]{4F81BD}{RAPPORT DE PROJET LEE}}\par}
\end{Center}\par


\par


\vspace{\baselineskip}
\setlength{\parskip}{9.96pt}
\setlength{\parskip}{0.0pt}
\begin{Center}
\textcolor[HTML]{4F81BD}{ }{\fontsize{28pt}{33.6pt}\selectfont \textbf{\textcolor[HTML]{4F81BD}{BROCANTE RECHERCHE}}\par}
\end{Center}\par


\vspace{\baselineskip}
\setlength{\parskip}{9.96pt}

\vspace{\baselineskip}

\vspace{\baselineskip}

\vspace{\baselineskip}
{\fontsize{18pt}{21.6pt}\selectfont Réalisé par :\par}\par

{\fontsize{18pt}{21.6pt}\selectfont Rouad ZEGGANE\par}\par


\vspace{\baselineskip}

\vspace{\baselineskip}
{\fontsize{18pt}{21.6pt}\selectfont Tuteurs du projet :\par}\par

{\fontsize{18pt}{21.6pt}\selectfont Pierre BOUDES\par}\par


\vspace{\baselineskip}

\vspace{\baselineskip}

\vspace{\baselineskip}

\vspace{\baselineskip}
\tab 
\vspace{\baselineskip}
\vspace{\baselineskip}

{\fontsize{18pt}{21.6pt}\selectfont \begin{Center}\textcolor[HTML]{4F81BD}{Master 2 Exploration Informatique des Données et Décisionnel}\end{Center}\par}\par

{\fontsize{18pt}{21.6pt}\selectfont
\begin{Center}
\textcolor[HTML]{4F81BD}{2018/2019}\end{Center}\par}\par




 %%%%%%%%%%%%  This Produces Table Of Contents %%%%%%%%%%%%%%


\vspace{\baselineskip}
\setlength{\parskip}{0.0pt}

\vspace{\baselineskip}

\vspace{\baselineskip}

\vspace{\baselineskip}

\vspace{\baselineskip}

\vspace{\baselineskip}

\vspace{\baselineskip}

\vspace{\baselineskip}

\vspace{\baselineskip}

\vspace{\baselineskip}

\vspace{\baselineskip}

\vspace{\baselineskip}

\vspace{\baselineskip}

\vspace{\baselineskip}

\vspace{\baselineskip}

\vspace{\baselineskip}

\vspace{\baselineskip}

\vspace{\baselineskip}

\vspace{\baselineskip}

\vspace{\baselineskip}

\vspace{\baselineskip}

\vspace{\baselineskip}

\vspace{\baselineskip}

\vspace{\baselineskip}

\vspace{\baselineskip}

\vspace{\baselineskip}

\vspace{\baselineskip}

\vspace{\baselineskip}

\vspace{\baselineskip}

\vspace{\baselineskip}

\vspace{\baselineskip}

\vspace{\baselineskip}

\vspace{\baselineskip}
\begin{enumerate}[label*={\fontsize{12pt}{12pt}\selectfont \textbf{\arabic*.}}]
	\item {\fontsize{18pt}{21.6pt}\selectfont \textbf{\textcolor[HTML]{4F81BD}{INTRODUCTION}}\par}\par


\vspace{\baselineskip}
\setlength{\parskip}{9.96pt}
\begin{justify}
Dans le cadre de notre projet LEE, j’ai choisi la réalisation d’une application web (Brocante recherche) parmi les sujets proposés par M. Pierre BOUDES.
\end{justify}\par

\setlength{\parskip}{0.0pt}
\begin{justify}
Le but est également d’apprendre à gérer un projet professionnel de la reformulation du cahier des charges jusqu’à sa réalisation complète en respectant les exigences du client. 
\end{justify}\par


\vspace{\baselineskip}
\begin{justify}
Vous verrez également dans ce projet une application directe des compétences acquises durant mon parcours universitaire notamment en ce qui concerne les langages J2EE, HTML, CSS et SQL.
\end{justify}\par

\begin{justify}
Le projet s’est déroulé en deux étapes :
\end{justify}\par

\begin{itemize}
	\item La définition du projet qui consiste à éditer un cahier des charges fonctionnel avec le client (le sujet proposé). Dans mon cas ce dernier étant déjà fait.\par

	\item La conception et la réalisation du projet : j’évoquerai dans cette partie le choix des outils et des langages utilisés pour la réalisation de l’application web, la base de données créée pour stocker les différentes informations, du développement proprement dit le codage, le design du site. Je parlerai aussi des résultats obtenus, des problèmes rencontrés lors de la conception et enfin je proposerai quelques idées d’amélioration du site.
\end{itemize}\par


\vspace{\baselineskip}
\begin{justify}
Dans les lignes qui suivent je vais détailler le développement de chacune de ces parties en étant le plus précis possible pour que le lecteur de ce rapport soit éclairé sur les différentes étapes de ce travail qui demande beaucoup de temps et qui tout de même est vraiment passionnant.
\end{justify}\par


\vspace{\baselineskip}

\vspace{\baselineskip}

\vspace{\baselineskip}

\vspace{\baselineskip}
\setlength{\parskip}{9.96pt}

\vspace{\baselineskip}

\vspace{\baselineskip}

\vspace{\baselineskip}

\vspace{\baselineskip}

\vspace{\baselineskip}

\vspace{\baselineskip}

\vspace{\baselineskip}

\vspace{\baselineskip}

\vspace{\baselineskip}
\vspace{\baselineskip}

\vspace{\baselineskip}
\setlength{\parskip}{5.04pt}
	\item {\fontsize{18pt}{21.6pt}\selectfont \textbf{\textcolor[HTML]{4F81BD}{PRESENTATION GENERALE DU SUJET}}\par}\par

\setlength{\parskip}{9.96pt}
\begin{justify}
Une petite annonce est un bref message inséré dans un journal, généralement dans le but d'annoncer la mise en vente ou en location d'un bien. Mais celles-ci sont généralement de courts textes comprenant de nombreuses abréviations, leur coût étant lié au nombre de lignes composant le message. Le développement d'internet a permis la diffusion des annonces à moindre coût tout en augmentant l'espace disponible pour chaque annonce. La plupart des petites annonces sont à présent accompagnées d'une photographie apportant une valeur informative importante sur le bien proposé.
\end{justify}\par

\begin{justify}
Mon logiciel permettra de gérer les annonces en offrant plusieurs fonctionnalités aux utilisateurs comme l’ajout d’une annonce à l’aide d’un formulaire ou la supprimer s’il est l’auteur de celle-ci. 
\end{justify}\par


\vspace{\baselineskip}
\setlength{\parskip}{5.04pt}
	\item {\fontsize{18pt}{21.6pt}\selectfont \textbf{\textcolor[HTML]{4F81BD}{ANALYSE APPROFONDIE}}\par}\par

\setlength{\parskip}{9.96pt}
\begin{enumerate}[label*={\fontsize{16pt}{16pt}\selectfont \textbf{\arabic*.}}]
	\item {\fontsize{16pt}{19.2pt}\selectfont \textbf{\textcolor[HTML]{4F81BD}{Analyse des besoins}}\par}\par


\vspace{\baselineskip}
L’objectif de ce logiciel est de gérer les annonces publiées par les déférents utilisateurs inscrits sur l’application. Pour la première utilisation\textbf{ }du logiciel, aucune configuration ou initialisation n’est requise. L’utilisateur va pouvoir communiquer avec le logiciel à l’aide des formulaires et des boutons. Le logiciel quant à lui, il communique avec la base de données à l’aide des requêtes SQL et en dirigeant les résultats vers l’utilisateur par des tableaux et des chaînes de caractères sur une interface simple. Il y a un type d’utilisateur qui réalise des objectifs primaires de ce logiciel qui est :\par


\vspace{\baselineskip}
\begin{itemize}
	\item \textbf{\textcolor[HTML]{4F81BD}{Client}}\par

\begin{justify}
Afin qu’un utilisateur puisse consulter, ajouter ou supprimer les annonces il doit s’inscrire sur l’application.
\end{justify}\par

	\item {\fontsize{14pt}{16.8pt}\selectfont \textbf{\textcolor[HTML]{4F81BD}{Les fonctionnalités du logiciel }}\par}\par


\vspace{\baselineskip}
	\item {\fontsize{14pt}{16.8pt}\selectfont \textbf{\textcolor[HTML]{4F81BD}{Gestion des annonces}}\par}
\end{itemize}\par

\begin{itemize}
	\item \textbf{Ajouter une annonce : }les clients peuvent ajouter des annonces à l’aide d’un formulaire comportant les champs suivant ; catégorie, titre, description, prix et photo.\par

	\item \textbf{Mes annonces :}{\fontsize{14pt}{16.8pt}\selectfont \textbf{ }les clients auront la possibilité de consulter ces propres annonces en cliquant sur le lien \textbf{mes annonces} et d’en supprimer à l’aide du\ bouton  \textbf{supprimer} en choisissant une de leurs\ annonces.  \par}\par

	\item \textbf{Accueil : }les clients auront la possibilité de consulter les différentes annonces proposées par d’autres clients une fois authentifier, comme ils peuvent rechercher\ les annonces via un formulaire de recherche en sélectionnant une catégorie et/ou écrivant le titre de l’annonce.  \textbf{ }
\end{itemize}\par

\setlength{\parskip}{0.0pt}

\end{enumerate}
\end{enumerate}\subsection{Les Outils de Développements }

\vspace{\baselineskip}
\begin{justify}
\textbf{Interface Graphique :}\textit{ }L’Interface Graphique est un dispositif de dialogue homme-machine, qui relie l’utilisateur à la base de données et permet de faciliter la réalisation des fonctionnalités voulues.
\end{justify}\par

\begin{justify}
\textbf{Configuration matérielle :}\textit{ }Ce logiciel fonctionnera sur un ordinateur bureautique Ou portable.
\end{justify}\par

\begin{justify}
\textbf{Configuration logicielle :}\textit{ }Ce logiciel fonctionnera sur tous les systèmes D’exploitation  avec un compilateur et bibliothèques java 1.6 ou supérieure.
\end{justify}\par

\begin{justify}
\textbf{Environnement de développement \textcolor[HTML]{D44501}{:}}\textcolor[HTML]{D44501}{ L'IDE Eclipse est un environnement de Développement intégré et disponible pour Windows, Mac, Linux, et Solaris. On a utilisé Eclipse IDE 4.4 qui offre une performance nettement améliorée et une Expérience de codage, avec des nouvelles fonctionnalités d’analyse statique de code dans l’éditeur Java. Eclipse constitue par ailleurs une plate-forme qui permet le développement d’applications spécifiques comme les applications java c’est pour cela on a l’utilisé.}
\end{justify}\par



%%%%%%%%%%%%%%%%%%%% Figure/Image No: 3 starts here %%%%%%%%%%%%%%%%%%%%

\begin{figure}[H]
\advance\leftskip 3.44in		\includegraphics[width=1.94in,height=1.69in]{./media/image4.png}
\end{figure}


%%%%%%%%%%%%%%%%%%%% Figure/Image No: 3 Ends here %%%%%%%%%%%%%%%%%%%%

\begin{justify}
\textcolor[HTML]{D44501}{\ \ \ \ \ \ \ \ \ \ \ \ \ \ \  }
\end{justify}\par



%%%%%%%%%%%%%%%%%%%% Figure/Image No: 4 starts here %%%%%%%%%%%%%%%%%%%%

\begin{figure}[H]
	\begin{Center}
		\includegraphics[width=2.03in,height=1.86in]{./media/image5.jpeg}
	\end{Center}
\end{figure}


%%%%%%%%%%%%%%%%%%%% Figure/Image No: 4 Ends here %%%%%%%%%%%%%%%%%%%%

\begin{justify}
\textcolor[HTML]{D44501}{\ \ \ \ \ \ \ \ \ \ \ \ \ \ \ \ \ \ \ \ \ \ \ \  }
\end{justify}\par

\begin{justify}
\textcolor[HTML]{D44501}{\ \ \ \ \ \ \ \ \ \ \ \ \ \ \ \ \ \ \ \ \ \ \ \ \ \ \ \ \ \ \ \ \ \ \ \ \ \ \ \  }
\end{justify}\par

\begin{justify}
\textcolor[HTML]{D44501}{\ \ \ \ \ \ \ \ \ \ \ \ \ \ \ \ \ \ \ \ \ \ \ \  }
\end{justify}\par


\vspace{\baselineskip}

\vspace{\baselineskip}
\begin{justify}
\textbf{Base de données :} Afin de relier le logiciel et stocker toutes les données on a utilisé le module MYSQL 5.6.22 ou supérieur.
\end{justify}\par

\begin{justify}
La base de données est composée de trois tables\textit{ }:
\end{justify}\par

\begin{justify}
\textcolor[HTML]{D44501}{- Table compte : sert à stocker les informations concernant les clients (identifiant, nom, prénom, mail, mot de passe et numéro de téléphone).}
\end{justify}\par

\begin{justify}
\textcolor[HTML]{D44501}{- Table catégorie : sert à stocker les informations de chaque catégorie (identifiant, titre).}
\end{justify}\par

\begin{justify}
\textcolor[HTML]{D44501}{- Table annonce : sert à stocker les annonces publiées par les clients (identifiant, identifiant client, titre, description, prix, photo, identifiant catégorie). }
\end{justify}\par



%%%%%%%%%%%%%%%%%%%% Figure/Image No: 5 starts here %%%%%%%%%%%%%%%%%%%%

\begin{figure}[H]
	\begin{Center}
		\includegraphics[width=2.7in,height=1.91in]{./media/image6.png}
	\end{Center}
\end{figure}


%%%%%%%%%%%%%%%%%%%% Figure/Image No: 5 Ends here %%%%%%%%%%%%%%%%%%%%

\begin{justify}
\textcolor[HTML]{D44501}{\ \ \ \ \ \ \ \ \ \ \ \ \ \ \ \ \ \ \ \ \ \ \ \ \ \ \ \ \ \ \ \ \ \ \ \ \ \ \ \ \ \ \  }
\end{justify}\par


\vspace{\baselineskip}
\begin{justify}
\textbf{Connexion :}\textit{ }Pour se connecter à la base de données MySQL j’ai utilisé l'API JDBC. JDBC est un ensemble de classes permettant de développer des applications capables de se connecter à des serveurs de base de données.
\end{justify}\par



%%%%%%%%%%%%%%%%%%%% Figure/Image No: 6 starts here %%%%%%%%%%%%%%%%%%%%

\begin{figure}[H]
\advance\leftskip 2.41in		\includegraphics[width=2.06in,height=2.04in]{./media/image7.jpeg}
\end{figure}


%%%%%%%%%%%%%%%%%%%% Figure/Image No: 6 Ends here %%%%%%%%%%%%%%%%%%%%

\par


\vspace{\baselineskip}
\setlength{\parskip}{9.96pt}
\ \ \ \ \ \  \par


\vspace{\baselineskip}

\vspace{\baselineskip}
\setlength{\parskip}{0.0pt}

\vspace{\baselineskip}

\vspace{\baselineskip}

\vspace{\baselineskip}

\vspace{\baselineskip}

\vspace{\baselineskip}

\vspace{\baselineskip}
	\item {\fontsize{18pt}{21.6pt}\selectfont \textbf{\textcolor[HTML]{4F81BD}{MANUEL}}\par}\par

\begin{enumerate}[label*={\fontsize{18pt}{18pt}\selectfont \textbf{\arabic*.}}]
	\item {\fontsize{14pt}{16.8pt}\selectfont \textbf{\textcolor[HTML]{4F81BD}{Manuel d’utilisation}}\par}\par

\begin{justify}
Après les phases, conception et modélisation j’ai développé les interfaces de l’application.
\end{justify}\par

\begin{justify}
La page ci-dessous représente la page d'accueil de notre application, à partir de cette interface, si l'internaute est un nouvel utilisateur il a la possibilité de créer son compte.
\end{justify}\par



%%%%%%%%%%%%%%%%%%%% Figure/Image No: 7 starts here %%%%%%%%%%%%%%%%%%%%

\begin{figure}[H]
\advance\leftskip 0.97in		\includegraphics[width=4.51in,height=2.85in]{./media/image8.png}
\end{figure}


%%%%%%%%%%%%%%%%%%%% Figure/Image No: 7 Ends here %%%%%%%%%%%%%%%%%%%%

\par



\begin{Center}
\textbf{\textcolor[HTML]{548DD4}{Figure 7 : Page d’accueil }}
\end{Center}\par


\vspace{\baselineskip}
À partir de cette interface, s'il est déjà inscrit, un utilisateur (Client) pourra se connecter. Il suffit d'entrer son login et son mot de passe\ et cliquer sur le bouton  connexion pour ouvrir sa session.\par

Bien évidemment, un nouvel utilisateur doit s'inscrire.\par

La figure ci-dessous présente l'interface de l'inscription d'un nouvel utilisateur\par



%%%%%%%%%%%%%%%%%%%% Figure/Image No: 8 starts here %%%%%%%%%%%%%%%%%%%%

\begin{figure}[H]
	\begin{Center}
		\includegraphics[width=6.3in,height=3.37in]{./media/image9.png}
	\end{Center}
\end{figure}


%%%%%%%%%%%%%%%%%%%% Figure/Image No: 8 Ends here %%%%%%%%%%%%%%%%%%%%

\par

\begin{Center}
\textbf{\textcolor[HTML]{548DD4}{Figure 8 : Page d’inscription}}
\end{Center}\par

\begin{justify}
Après validation de ses informations et création de son profil, le client sera redirigé vers son profil. L’image ci-dessous présente l'interface de son profil.
\end{justify}\par




%%%%%%%%%%%%%%%%%%%% Figure/Image No: 9 starts here %%%%%%%%%%%%%%%%%%%%

\begin{figure}[H]
	\begin{Center}
		\includegraphics[width=6.19in,height=2.9in]{./media/image10.png}
	\end{Center}
\end{figure}


%%%%%%%%%%%%%%%%%%%% Figure/Image No: 9 Ends here %%%%%%%%%%%%%%%%%%%%

\begin{Center}
\textbf{\textcolor[HTML]{548DD4}{Figure 9 : profil d’un client}}
\end{Center}\par


\vspace{\baselineskip}


%%%%%%%%%%%%%%%%%%%% Figure/Image No: 10 starts here %%%%%%%%%%%%%%%%%%%%




%%%%%%%%%%%%%%%%%%%% Figure/Image No: 10 Ends here %%%%%%%%%%%%%%%%%%%%

\begin{justify}
A partir de cette interface le client aura la possibilité d’ajouter des annonces. La figure ci-dessous présente l'interface ajouter une annonce
\end{justify}\par

\begin{figure}[H]
\begin{Center}	
    \includegraphics[width=6.3in,height=3.62in]{./media/image11.png}
\end{Center}  
\end{figure}

\begin{Center}
\textbf{\textcolor[HTML]{548DD4}{Figure 10 : Page ajouter une annonce}}
\end{Center}\par


\vspace{\baselineskip}

\vspace{\baselineskip}
{\fontsize{14pt}{16.8pt}\selectfont \textbf{\textcolor[HTML]{4F81BD}{4.2\ \ \  Compilation du projet}}\par}\par

\textbf{\textcolor[HTML]{595959}{Etape 1 :}}\textcolor[HTML]{595959}{ Lancer le serveur mamp ou mysql }\par

\textbf{\textcolor[HTML]{595959}{Etape 2 : }}\textcolor[HTML]{595959}{Lancer le serveur tomcat dans eclipse }\par

\textbf{\textcolor[HTML]{595959}{Etape 3 :}}\textcolor[HTML]{595959}{ Sélectionner le projet et cliquer sur run\ \  }\par


\vspace{\baselineskip}
\setlength{\parskip}{9.96pt}

\end{enumerate}
\printbibliography
\end{document}